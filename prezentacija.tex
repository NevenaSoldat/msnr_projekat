\documentclass{beamer}

\usepackage[utf8]{inputenc}
\usepackage{graphicx}
\usetheme{Copenhagen}
\usecolortheme{default}
\usepackage{media9}
\renewcommand{\figurename}{Slika}

%Information to be included in the title page:
\title[Optimizacija rojem čestica]{Optimizacija rojem čestica}
\author{Nevena Soldat, Tijana Živković, Ana Miloradović, Milena Kurtić}
\institute{Matematički fakultet, Univerzitet u Beogradu}
\date{\today}



\begin{document}
\frame{\titlepage}

\begin{frame}
\frametitle{Pregled}
\begin{itemize}
    \item{Uvod u optimizaciju rojem čestica}
    \item{Osnovni algoritam}
    \item{Varijacije parametara}
    \item{Primena}
    \item{Topologije}
    \item{Literatura}
\end{itemize}

\end{frame}

\begin{frame}{Uvod}

Šta je optimizacija rojem čestica?
\begin{itemize}
    \item optimizaciona tehnika zasnovana na inteligentnom ponašanju nekih organizama, kao što su insekti, ptice i ribe

\end{itemize}

Nastanak:
\begin{itemize}
    \item Eugene Marais - The Soul of the White Ant (1926)
    \item Marco Dorigo - ponašanje kolonije mrava (1990-ih) 
    \item Eberhart i Kennedy - algoritam optimizacije rojem čestica (1995)
\end{itemize}

Algoritam za optimizaciju rojem čestica je otkriven sasvim slučajno, pri pokušaju da se na računaru simulira kretanje jata ptica.
\end{frame}

\begin{frame}{Osnovni algoritam}
Osnovni koncepti:
\begin{itemize}
    \item čestice se kreću kroz višedimenzioni prostor pretrage
    \item svaka čestica predstavlja jedno moguće rešenje
\end{itemize}

Neka je $x_i(t)$ pozicija čestice \textit{i} u trenutku \textit{t}. Pozicija se menja dodavanjem brzine $v_i(t)$ na trenutku poziciju:
\[ x_i(t+1) = x_i(t) + v_i(t+1) \]
Brzina se računa kao:
\[ v_i(t+1) = v_i(t) + c_1r_1(p_i(t) - x_i(t)) + c_2r_2(p_g(t) - x_i(t))\] gde su:

\begin{itemize}
    \item $p_i(t)$ - najbolja pozicija koju je čestica \textit{i} pronašla do trenutka \textit{t}
    \item $p_g(t)$ - najbolja pozicija u čitavom roju do trenutka \textit{t}
    \item $r_1, r_2$ - nasumične vrednosti iz \textit{U}[0,1]
    \item $c_1, c_2$ - konstante
\end{itemize}

\end{frame}


\begin{frame}{Komponente brzine}
    \begin{figure}[htp]
    \centering
    \includegraphics[scale = 1.2]{velocity.png}
    \caption{Komponente brzine}
    \label{fig:velocity}
\end{figure}

\begin{itemize}
    \item \textbf{moment} - prethodno stanje brzine
    \item \textbf{kognitivna} komponenta - tendencija vraćanja u lično najbolje
    \item \textbf{socijalna} komponenta - tendencija ka kretanju ka naboljem globalnom
\end{itemize}
\end{frame}

\end{document} 
